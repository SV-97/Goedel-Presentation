\subsection{Gödel's ontological proof}
\begin{frame}
    \frametitle{Gödel - the one who unified mathematics and philosophy}
    \begin{itemize}
        \item around 1941: first version of \textbf{Gödel's ontological proof}(formal argument for the existence of god)
        \pause \item inspired by Gottfried Leibniz: one of the most important logicians and natural philosophers of the Enlightenment
        \pause \item includes fourteen points of his philosophical beliefs
        \begin{itemize}
            \pause \item 4. There are other worlds and rational beings of a different and higher kind.
            \pause \item 5. The world in which we live in is not the only one in which we shall live or have lived.
            \pause \item 13. There is a scientific philosophy and theology, which deals with concepts of the highest abstractness and this is also most highly fruitful for science.
            \pause \item 14. Religions are for the most part bad - but religion is not.
        \end{itemize}
    \end{itemize}
\end{frame}

\begin{frame}
    \frametitle{Gödel - the one who unified mathematics and philosophy}
    \textbf{Proof}
    \begin{itemize}
        \item uses modal logic
        \item difficult to understand and the notation hard to read
        \item excerpt of the proof: 
        \small\begin{align*}
            \text{Ax. 1 }& (P(\phi) \wedge \Box \forall x : \phi(x) \implies \psi(x)) \implies P(\psi)
            \\ \text{Ax. 2 }& P(\neg\phi) \iff \neg P(\phi)
            \\ \text{Th. 1 }& P(\phi) \implies \diamond \exists x : \phi(x)
            \\ \vdots
            \\ \text{Df. 2 }& \phi \text{ ess } x \iff \phi(x) \wedge \forall \psi : (\psi(x) \implies \Box \forall y: (\phi(y) \implies \psi(y)))
            \\ \text{Ax. 4 }& P(\phi) \implies \Box P(\phi)
            \\ \vdots
            \\ \text{Th. 4 }& \Box \exists x : G(x)
        \end{align*}
    \end{itemize}

\end{frame}

\begin{frame}
    \frametitle{Gödel - the one who unified mathematics and philosophy}
    \textbf{Rough summary of proof:}
    \emph{Existence of ''something'', that is the unification of all that is positive.}
    
    \pause \textbf{Religious views:}
    He was himself very religious, but not a member of any religious congregation.

    \pause $\implies$ lots of criticism, but overall ground-breaking and pioneering for all future philosophers and logicians
\end{frame}

% \subsection{Work on the continuum hypothesis}
% \begin{frame}
%     \frametitle{The continuum hypothesis}
%     \begin{center}
%         \Huge$$
%         2^{\aleph_0} = \aleph_1
%         $$
%     \end{center}
% \end{frame}