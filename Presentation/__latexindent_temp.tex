\documentclass[10pt]{beamer}

\usetheme[progressbar=frametitle]{metropolis}

%\usepackage[T1]{fontenc}
\usepackage[english]{babel}
\usepackage{newunicodechar}
\usepackage[utf8]{inputenc}

\usepackage{subcaption}
\usepackage{adjustbox}
\usepackage{booktabs}
\usepackage[scale=2]{ccicons}

% For pseudo codes
\usepackage{algorithm}
\usepackage[noend]{algpseudocode}
\makeatletter
\def\BState{\State\hskip-\ALG@thistlm}
\makeatother
%

\usepackage{multirow}
\usepackage[none]{hyphenat}
\usepackage{textcomp}
\usepackage{gensymb}
\sloppy 
%\usebackgroundtemplate


\usepackage{pgfplots}
\usepgfplotslibrary{dateplot}

\usepackage{xspace}
\newcommand{\themename}{\textbf{\textsc{metropolis}}\xspace}

\usepackage{graphicx}
\graphicspath{ {./img/} }

\usepackage{fancyhdr}

\usepackage{xargs}

\usepackage{chronology} % timeline
\usepackage{changepage} % adjustwidth environment
% \usepackage{stmaryrd} % lightning

\setbeamercolor{background canvas}{bg=white!20}

\title{Kurt Gödel}
\subtitle{Mathematician, Logician, Philosopher}
\date{03.07.2020}
\author{Anna Frank, Stefan Volz}
\institute{University of Applied Sciences Würzburg-Schweinfurt\\Faculty of Applied Natural Sciences and Humanities\\B.Sc. Industrial Mathematics\\English for industrial mathematicians}
\titlegraphic{\small\center
\flushright\includegraphics[height=1.0cm]{FHWS}}

\rhead{\includegraphics[width=1.5cm]{FHWS}}

\logo{\includegraphics[width=1cm]{FHWS}\hfill}
\newcommand{\nologo}{\setbeamertemplate{logo}{}} % command to set the logo to nothing
\newcommand{\congress}{\hspace{1cm}\small{Frank, Volz}} % 1cm offset to compensate for 1cm logo

% footer
\makeatletter
\setbeamertemplate{footline}
{
  \leavevmode%
  \hbox{%

  \begin{beamercolorbox}[wd=.9\paperwidth,ht=2.25ex,dp=1ex,center]{institute in head/foot}%
    \usebeamerfont{abstract}%
    \congress
  \end{beamercolorbox}%

  \begin{beamercolorbox}[wd=.1\paperwidth,ht=2.25ex,dp=2ex,center]{institute in head/foot}%
    \usebeamerfont{abstract} 
    \insertframenumber{} / \inserttotalframenumber 
  \end{beamercolorbox}}% 
  
  \begin{beamercolorbox}[wd=\paperwidth,ht=2.5ex,dp=1.125ex]{palette quaternary}%
    \insertsubsectionnavigationhorizontal{\paperwidth}{}{\hskip0pt plus1filll}
    \end{beamercolorbox}%
}

% remove indents at paragraphs
\setlength{\parindent}{0pt}

% remove indentation of first line of quotation
% from: https://tex.stackexchange.com/questions/149868/define-quotation-environment-without-indentation-of-the-first-paragraph
\renewenvironment{quotation}
  {\list{}{\listparindent 1.5em%
          %\itemindent    \listparindent
          %\rightmargin \leftmargin
          \parsep        \z@ \@plus\p@}%
  \item\relax}
{\endlist}

%\AtBeginSection[]
%{
%  \begin{frame}
%  \frametitle{Overview}
%  \setbeamertemplate{section in toc}[sections numbered]
%  \tableofcontents[currentsection, currentsubsection, 
%  hideothersubsections, 
%  sectionstyle=show/shaded,]
%  \end{frame}
%}

%\setbeamertemplate{footline}[text line]{%
%  \parbox{\linewidth}{\vspace*{-8pt}some text\hfill\insertshortauthor\hfill\insertpagenumber}}
%\setbeamertemplate{navigation symbols}{}

\setcounter{tocdepth}{1} % only show sections in TOC

\makeatother
\begin{document}

\maketitle
%%%%%%%%%%%%%%%%%%%%%%%%%%%%%%%%%%%%%%%%%%%%%%%%%%%%%%%%%%%%%%
\section{Allgemeines zur FHWS} 

\begin{frame}[fragile]{Allgemeines zur FHWS}
  \begin{columns}
    \begin{column}{.5\textwidth}
      \begin{figure}[ht]  
        \adjustimage{width=.95\textwidth,left}{Rundbau.png}
      \end{figure}
      \textbf{Anmeldezeitraum}\\
      Fr. 01.05.2020 bis \\Mi. 15.07.2020
    \end{column}
  
    \begin{column}{.5\textwidth}
      \textbf{Zulassungsvorraussetzungen/\\Hochschulzugangsmöglichkeiten}
      \begin{itemize}
        \item Allgemeine Hochschulreife
        \item Fachhochschulreife
        \item Meister oder staatl. gepr. Techniker
        \item Kein NC bei technischen Studiengängen
      \end{itemize}
      \textbf{Zur FHWS} (in Schweinfurt)\\
      \begin{itemize}
        \item 2858 Studierende
        \item 86 Professoren
        \item 24 Bachelor- und 18 Masterstudiengänge
      \end{itemize}
    \end{column}
  \end{columns}
\end{frame}

\section{Was ist Technomathematik?}

\begin{frame}[fragile]{Was ist Technomathematik?}
  \begin{figure}[ht]  
    \adjustimage{height=.7\textheight, center}{Venn.jpg}
  \end{figure}
\end{frame}


\begin{frame}[fragile]{Studieninhalte}
  \begin{columns}
    \fbox{
      \begin{column}{.3\textwidth}
        \textbf{Mathematik}\\
        Grundstudium \\
        \begin{itemize}
          \item Lineare Algebra
          \item Analysis
          \item Statistik
        \end{itemize}
        Hauptstudium \\
        \begin{itemize}
          \item Numerik
          \item Optimierung
        \end{itemize}
      \end{column}
    }
    \pause
    \fbox{
      \begin{column}{.4\textwidth}
        \textbf{Technik}\\
        Klassisch \\
        \begin{itemize}
          \item Physik
          \item Elektrotechnik
          \item Maschinenbau
          \item Simulationspraktikum
          \item Konstruktionselemente und Festigkeitslehre
        \end{itemize}
        Simulation im Maschinenbau \\
        \begin{itemize}
          \item Simulationstechnik
          \item Ingenieurprojekt
        \end{itemize}
      \end{column}
    }
    \pause
    \fbox{
      \begin{column}{.3\textwidth}
        \textbf{Informatik}\\
        \begin{itemize}
          \item Grundlagen der Programmierung
          \item Objektorientierte Programmierung
          \item Datenbanken
          \item Maschinelles Lernen
        \end{itemize}
      \end{column}
    }
  \end{columns}
  Außerdem mögliche Schwerpunktswahl (Data Science, Informatik oder frei gewählte Fächer) ab dem 5. Semester
\end{frame}

\begin{frame}[fragile]{Allgemeines zum Studium}
  \begin{columns}
    \begin{column}{.6\textwidth}
      \begin{itemize}
        \item Sieben Semester Regelstudienzeit
        \item Kleine Klassen (i.d.R. ca. 20 Studierende) $\implies$ sehr gute Betreuung
        \item Abschluss: Bachelor of Science
        \item Großer Selbststudienanteil
      \end{itemize}
    \end{column}
    \begin{column}{.4\textwidth}
      \begin{figure}[ht]
        \adjustimage{width=\textwidth, right}{Students_Grass.jpg}
      \end{figure}
    \end{column}
  \end{columns}
\end{frame}

\section{Was macht ein Technomathematiker eigentlich?}

\begin{frame}[fragile]{Fallbeispiel 1: Simulation}
  \begin{columns}
    \begin{column}{.5\textwidth}
      \begin{figure}[ht]
        \adjustimage{width=\textwidth, center}{FEM_Aircraft.jpg}
        \caption{Spannungssimulation an einem Flugzeugteil}
      \end{figure}
    \end{column}
    \begin{column}{.5\textwidth}
      \begin{figure}[ht]
        \adjustimage{width=\textwidth, center}{FEM_Cow.png}
        \caption{Aerodynamik eines Bauteils}
      \end{figure}
    \end{column}
  \end{columns}
\end{frame}

\section{Berufsmöglichkeiten}

\begin{frame}[fragile]{Berufsmöglichkeiten}
  \begin{itemize}
    \item Statistik
    \item Simulation
    \item Finanzen
    \item Logistik
    \item Softwareentwicklung
    \item Automatisierungstechnik (Robotik)
    \item Qualitätssicherung
  \end{itemize}
\end{frame}

\begin{frame}[fragile]{Zusammenfassung}
  \begin{itemize}
    \item $Technomathematik = Mathematik + Technik + Informatik$
    \item Verschiedene mögliche Schwerpunkte
      \begin{itemize}
        \item Informationstechnologie
        \item Data Science
        \item Simulation im Maschinenbau
        \item Individuelle Zusammenstellung
      \end{itemize}
    \item Vielfältige Berufs- und Weiterbildungsmöglichkeiten
      \begin{itemize}
        \item Master an FHWS möglich: Angewandte Mathematik und Physik
      \end{itemize}
    \item Sieben Semester
  \end{itemize}
\end{frame}


{\nologo
%\begin{frame}[allowframebreaks]
%        \frametitle{References}
%        \bibliographystyle{abntex2-alf}
%        \bibliography{ref.bib}
%\end{frame}
%%%%%%%%%%%%%%%%%%%%%%
%\newcommand{\nologo}{\setbeamertemplate{logo}{}} % command to set the logo to nothing

  \begin{frame}[standout]
  Danke für eure Aufmerksamkeit \\ Fragen?
  \end{frame}
}
\end{document}
